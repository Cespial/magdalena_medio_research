\documentclass[11pt]{letter}
\usepackage[utf8]{inputenc}
\usepackage[T1]{fontenc}
\usepackage[margin=2.5cm]{geometry}
\usepackage{amsmath}
\usepackage{hyperref}

\signature{Cristian Espinal Maya\\School of Finance, Economics and Government\\Universidad EAFIT, Medell\'{i}n, Colombia}
\address{Cristian Espinal Maya\\School of Finance, Economics and Government\\Universidad EAFIT\\Medell\'{i}n 050022, Colombia\\cjespinalm@eafit.edu.co}
\date{February 16, 2026}

\begin{document}

\begin{letter}{Editor-in-Chief\\Applied Geography (Elsevier)}

\opening{Dear Editor,}

We are pleased to submit our manuscript entitled \textbf{``Accelerating Deforestation After Peace: Land Use Transitions and Carbon Loss in Colombia's Magdalena Medio (2012--2024)''} for consideration for publication in \textit{Applied Geography}.

This study addresses one of the most pressing applied-geographic questions in the post-conflict Global South: how do land use and land cover change (LULCC) trajectories respond when armed conflict subsides and governance vacuums emerge? Using Magdalena Medio---a 3.7-million-hectare region at the intersection of agrarian expansion, extractive industries, and Colombia's 2016 peace agreement---as a case study, we quantify twelve years of land use transitions, their carbon consequences, and their spatially varying drivers. We believe the manuscript's integration of remote sensing, ecosystem service valuation, and spatial regression analysis aligns closely with \textit{Applied Geography}'s scope in applied spatial research that bridges environmental science and policy.

\textbf{Key contributions:}

\begin{enumerate}
\item \textbf{First comprehensive post-conflict LULCC assessment of Magdalena Medio:} We produce four multi-class land cover maps (2013, 2016, 2020, 2024) from harmonized Landsat~8/9 and Sentinel-2 composites processed in Google Earth Engine, with adjusted overall accuracies of 59--63\%. Using Olofsson et al.\ (2014) stratified area estimation with full uncertainty propagation, we document a forest decline of ${\sim}$21\% ($\pm$18\%), from ${\sim}$2,021,000~ha to ${\sim}$1,591,000~ha.

\item \textbf{Multi-product cross-validation:} We validate our forest-change estimates against four independent datasets, confirming the directional consistency of forest decline across five products. This multi-product convergence strengthens confidence in the detected trends despite moderate per-class accuracies.

\item \textbf{Carbon stock quantification with rigorous uncertainty:} Applying IPCC Tier~2 carbon pools calibrated for Colombian ecosystems, we estimate a net carbon loss of ${\sim}$61~Tg~C (${\sim}$222~Mt~CO$_2$), with stocks declining from $435 \pm 76$ to $374 \pm 74$~Tg~C. Monte Carlo error propagation under correlated classification uncertainty yields $P(\text{net loss}) = 0.953$.

\item \textbf{Spatial driver analysis:} Getis-Ord Gi* hotspot analysis identifies 648 statistically significant deforestation cells at the 99\% confidence level. Geographically Weighted Regression (GWR; $R^2 = 0.188$) substantially outperforms Ordinary Least Squares ($R^2 = 0.121$; $\Delta$AIC $= 4{,}711$), revealing spatially varying deforestation drivers along river corridors---a pattern invisible to global models and directly relevant to place-based policy design.

\item \textbf{Post-conflict environmental governance:} Our findings contribute to the understanding of post-conflict environmental dynamics with direct relevance to Sustainable Development Goals~6 (Clean Water), 13 (Climate Action), 15 (Life on Land), and 16 (Peace, Justice and Strong Institutions), and to the broader literature on the environmental dimensions of peacebuilding.

\item \textbf{Reproducibility:} The entire analytical pipeline is implemented in Google Earth Engine, ensuring transparency and transferability to other post-conflict landscapes.
\end{enumerate}

The manuscript has not been previously published. A preprint version has been deposited on SSRN to ensure early dissemination; the preprint is not a peer-reviewed publication and does not constitute prior publication. The manuscript is not under consideration by any other journal. All authors have read and approved the submitted version.

We appreciate your consideration and look forward to your response.

\closing{Sincerely,}

\end{letter}
\end{document}
