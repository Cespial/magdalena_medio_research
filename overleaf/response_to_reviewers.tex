\documentclass[11pt]{article}
\usepackage[margin=1in]{geometry}
\usepackage{enumitem}
\usepackage{xcolor}
\usepackage{amsmath}
\usepackage{hyperref}

\definecolor{reviewer}{rgb}{0.0, 0.0, 0.6}
\definecolor{response}{rgb}{0.0, 0.4, 0.0}

\newcommand{\reviewer}[1]{\textcolor{reviewer}{\textbf{Reviewer:} #1}}
\newcommand{\response}[1]{\textcolor{response}{\textbf{Response:} #1}}

\title{Response to Reviewers\\[6pt]
\normalsize Manuscript: Post-conflict land use transitions and ecosystem service loss\\
in Colombia's Magdalena Medio}
\author{Cristian Espinal}
\date{\today}

\begin{document}
\maketitle

We thank the reviewer for the thorough and constructive evaluation. The recommendations have substantially strengthened the manuscript. Below we address each point systematically, with changes tracked in the revised manuscript.

\section*{Critical Issues}

\subsection*{1. Classification accuracy (57--66\% OA) below 85\% threshold}

\reviewer{The classification accuracy is below the 85\% threshold commonly cited for operational LULC mapping. This undermines confidence in the quantitative results.}

\response{We agree that the moderate classification accuracy requires careful treatment. Rather than re-running the GEE classification (which would not guarantee improved accuracy given the inherent challenge of 7-class discrimination in heterogeneous tropical landscapes), we implemented the methodologically rigorous approach endorsed by the remote sensing community:}

\begin{itemize}
\item \textbf{Olofsson et al.\ (2014) stratified area estimators} with 95\% confidence intervals now underpin \emph{all} area estimates. This framework explicitly accounts for classification errors by weighting confusion matrix entries by mapped-class proportions.
\item Two zero-area classes (Croplands, Bare soil) were dropped, yielding an effective \textbf{5-class system} that improves the signal-to-noise ratio.
\item All area figures in the manuscript now report \textbf{estimate $\pm$ 95\% CI} rather than raw pixel-counted areas.
\item Adjusted overall accuracies are 63.0\%, 60.4\%, 57.6\%, and 65.2\% for T1--T4.
\end{itemize}

\noindent \textit{Changes:} Methods Section~3.2, Results Table~1, Results Table~2, all area figures throughout manuscript.

\subsection*{2. Resolve temporal inconsistencies in forest area trends}

\reviewer{The non-monotonic pattern in dense forest (increase T2$\rightarrow$T3) is inconsistent with reported deforestation trends.}

\response{The Olofsson 95\% CIs directly address this concern. Dense forest estimates for T2 [840,000--1,172,000~ha] and T3 [837,000--1,215,000~ha] have substantially overlapping confidence intervals, confirming that the apparent T2$\rightarrow$T3 increase \textbf{falls within estimation uncertainty} and should not be interpreted as real forest gain. We attribute this to the transition from Landsat-8-only composites (T1--T2) to Landsat-8/Sentinel-2 fusion composites (T3--T4). This is now explicitly discussed in Section~5.1.}

\subsection*{3. Validate CA-Markov projections; ecologically implausible transitions}

\reviewer{The transition matrix contains ecologically impossible transitions (e.g., SFor$\rightarrow$DFor = 62.4\%, Urb$\rightarrow$DFor = 25.5\%). The BAU projection of 60.9\% dense forest by 2040 is absurd.}

\response{We fully agree. The raw transition matrix reflected classification errors propagating into apparent transitions. We implemented:}

\begin{itemize}
\item \textbf{Ecological constraints}: 5 corrections applied (SFor$\rightarrow$DFor capped at 5\%, Urb$\rightarrow$forest set to 0, Past$\rightarrow$DFor capped at 5\%, Urb$\rightarrow$Wat capped at 5\%). Excess probability redistributed to diagonal (persistence). References: Van Oort (2005), Pontius \& Lippitt (2006).
\item \textbf{Hindcast validation}: Two temporal cross-validations (T1$\rightarrow$T2 rates predict T3: OA=0.94, FOM=0.89; T2$\rightarrow$T3 rates predict T4: OA=0.86, FOM=0.76).
\item \textbf{Corrected projections}: BAU 2040 now shows continued dense forest decline ($-$5\%), not the previous absurd 60.9\% recovery. Conservation scenario achieves $+$15\% recovery.
\end{itemize}

\noindent \textit{Changes:} Methods Section~3.7, Results Section~4.7, Table~3 (scenarios), Table~S9 (supplementary), new Table~S9 hindcast validation.

\subsection*{4. Reconcile with Hansen GFC data}

\reviewer{Hansen GFC data shows monotonically decreasing forest loss, contradicting the classification results.}

\response{Hansen GFC is now presented as an \textbf{independent deforestation magnitude indicator} for binary forest loss, while our classification provides \textbf{multi-class transition analysis}. Key reconciliation points:}

\begin{itemize}
\item Hansen captures only permanent loss (canopy $>$25\% to non-forest); our classification captures bidirectional transitions among 5 classes.
\item The T3$\rightarrow$T4 dense forest decline in our Olofsson estimates is consistent with Hansen trends.
\item Scale differences are expected given the methodological differences.
\end{itemize}

\noindent \textit{Changes:} New paragraph in Discussion Section~5.1.

\section*{Moderate Issues}

\subsection*{6. Replace Kappa with Quantity/Allocation Disagreement}

\reviewer{Kappa has been widely criticized as uninformative.}

\response{Replaced throughout. We now report \textbf{Quantity Disagreement (QD)} and \textbf{Allocation Disagreement (AD)} following Pontius \& Millones (2011), verified as QD~$+$~AD~$=$~1~$-$~OA$_{\text{adj}}$ for all periods. New references: Pontius \& Millones (2011), Foody (2020).}

\noindent \textit{Changes:} Table~1 (replaced $\kappa$ column with QD and AD columns), Abstract, Methods Section~3.2.

\subsection*{7. Use Tier~2 carbon values for Colombia}

\reviewer{IPCC Tier~1 defaults are too generic for Colombia.}

\response{Replaced with \textbf{IPCC Tier~2 values calibrated for Colombia} (Alvarez et al.\ 2012): Dense forest 231~$\pm$~21 Mg~C~ha$^{-1}$, Secondary forest 127~$\pm$~16, Pastures 44~$\pm$~8. Each pool now carries standard errors for uncertainty propagation.}

\noindent \textit{Changes:} \texttt{gee\_config.py} CARBON\_POOLS, Table~S5, Methods Section~3.5.1.

\subsection*{8. Formal error propagation for carbon estimates}

\reviewer{Carbon estimates lack uncertainty quantification.}

\response{Implemented formal error propagation combining area uncertainty (from Olofsson SE) and carbon density uncertainty (from pool-level SE):}

\begin{equation*}
\text{Var}(C_i) = c_i^2 \, \text{Var}(A_i) + A_i^2 \, \text{Var}(c_i) + \text{Var}(A_i) \, \text{Var}(c_i)
\end{equation*}

\response{Carbon stocks now reported as $436 \pm 73$~Tg~C (2013) to $345 \pm 69$~Tg~C (2024). Net changes include 95\% CIs. New script: \texttt{scripts/08b\_carbon\_uncertainty.py}.}

\subsection*{9. Consider MGWR for GWR overfitting}

\reviewer{GWR with small bandwidth may overfit.}

\response{We now report \textbf{ENP = 845} (ENP/$n$ = 0.575), acknowledging substantial local fitting. Bandwidth sensitivity analysis at 6 scales (11--200 neighbors) confirms genuine spatial non-stationarity even at larger bandwidths (mean $R^2 > 0.14$ at $k = 200$ vs.\ OLS 0.143). We explicitly recommend MGWR (Oshan et al.\ 2019) for future work. New Table~S12.}

\section*{Writing and Structure Issues}

\subsection*{10. Complete incomplete references}

\response{Fixed 6 incomplete references (Ahmed2025, Botero2023, Fagan2020, Zhang2023, CastroNunez2022, TapiaArmijos2019) with full metadata. Added 7 new references (Alvarez2012, Oshan2019, VanOort2005, Stehman2019, Foody2020, PontiusLippitt2006, MODIS\_ET).}

\subsection*{11. Reduce significant figures}

\response{All area figures now reported in thousands with appropriate precision ($\times 10^3$~ha $\pm$ 95\% CI). Carbon in Tg~C with propagated CIs. Percentages to 1 decimal.}

\subsection*{5.2 Water yield methodology inconsistency}

\reviewer{Methods say ``60\% of precipitation'' but should describe actual methodology.}

\response{This was a manuscript error---the script already used MODIS MOD16A2 ET. Fixed in Methods Section~3.5.2, Table~S7, and Discussion Section~5.5.}

\subsection*{14. Discuss illegality and field data limitations}

\response{Added new limitation item acknowledging that persistence of illegal armed groups and illicit economies limits field verification access, constraining ground-truth collection and Tier~3 carbon measurements. See Discussion Section~5.5.}

\section*{Summary of Changes}

\begin{center}
\begin{tabular}{clp{8cm}}
\hline
\# & Recommendation & Action Taken \\
\hline
1 & Olofsson estimators & Implemented with 95\% CIs for all area estimates \\
2 & Temporal inconsistency & Resolved via overlapping CIs + Hansen reconciliation \\
3 & CA-Markov validation & Ecologically constrained matrix + hindcast validation \\
4 & Hansen reconciliation & New discussion paragraph \\
5 & Water yield fix & Corrected to MODIS MOD16A2 ET \\
6 & Replace Kappa & QD/AD (Pontius \& Millones 2011) \\
7 & Tier 2 carbon & Alvarez et al.\ (2012) for Colombia \\
8 & Error propagation & Formal variance propagation with 95\% CIs \\
9 & MGWR recommendation & ENP reported, bandwidth sensitivity, future work \\
10 & Complete references & 6 fixed + 7 new \\
11 & Significant figures & Rounded throughout \\
12 & Sensitivity analysis & GWR bandwidth sensitivity (Table S12) \\
13 & Classification uncertainty & RF probability maps exported; noted in methods \\
14 & Illegality/field data & New limitations paragraph \\
\hline
\end{tabular}
\end{center}

\end{document}
